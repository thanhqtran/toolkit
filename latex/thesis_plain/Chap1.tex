\numberwithin{equation}{chapter}
\chapter{Chapter name}

\label{chap1}

%enable chapter abstract

\begin{singlespace}
\begin{chapabstract}
    \noindent This chapter explores the relationship between declining birth rates and economic growth within an overlapping generations model that incorporates endogenous fertility, tertiary education choices, and R\&D activities. As technological progress increases the returns to education, individuals are incentivized to pursue higher education. However, if the opportunity cost of child-rearing is sufficiently high and access to tertiary education remains too open, an oversupply of highly educated individuals with subreplacement fertility can push the economy toward prolonged stagnation or population decline. In the stagnation scenario, the economy maintains a constant population size and welfare level without technological progress, whereas in the decline scenario, the population asymptotically approaches zero, with welfare decreasing persistently. Our findings suggest that restricting access to tertiary education could correct the long-term trajectory and restore balanced growth for future generations. Nevertheless, implementing such a policy requires careful consideration as it may impose welfare loss on the current population.
\end{chapabstract}

	\begin{keywords}
		education choice, fertility decline, overlapping generations, R\&D growth
	\end{keywords}
	\begin{jelclass}
		E13, J13, J14, J22, J24, O11
	\end{jelclass}

\end{singlespace}
\clearpage



\section{Introduction}
lorem

\section{Model}

Ipsum
