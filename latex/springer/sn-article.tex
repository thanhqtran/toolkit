%Version 2.1 April 2023
% See section 11 of the User Manual for version history
%
%%%%%%%%%%%%%%%%%%%%%%%%%%%%%%%%%%%%%%%%%%%%%%%%%%%%%%%%%%%%%%%%%%%%%%
%%                                                                 %%
%% Please do not use \input{...} to include other tex files.       %%
%% Submit your LaTeX manuscript as one .tex document.              %%
%%                                                                 %%
%% All additional figures and files should be attached             %%
%% separately and not embedded in the \TeX\ document itself.       %%
%%                                                                 %%
%%%%%%%%%%%%%%%%%%%%%%%%%%%%%%%%%%%%%%%%%%%%%%%%%%%%%%%%%%%%%%%%%%%%%

%%\documentclass[referee,sn-basic]{sn-jnl}% referee option is meant for double line spacing

%%=======================================================%%
%% to print line numbers in the margin use lineno option %%
%%=======================================================%%

%%\documentclass[lineno,sn-basic]{sn-jnl}% Basic Springer Nature Reference Style/Chemistry Reference Style

%%======================================================%%
%% to compile with pdflatex/xelatex use pdflatex option %%
%%======================================================%%

%%\documentclass[pdflatex,sn-apa]{sn-jnl}% Basic Springer Nature Reference Style/Chemistry Reference Style


%%Note: the following reference styles support Namedate and Numbered referencing. By default the style follows the most common style. To switch between the options you can add or remove “Numbered” in the optional parenthesis. 
%%The option is available for: sn-basic.bst, sn-vancouver.bst, sn-chicago.bst, sn-mathphys.bst. %  

%%\documentclass[sn-nature]{sn-jnl}% Style for submissions to Nature Portfolio journals
%%\documentclass[sn-basic]{sn-jnl}% Basic Springer Nature Reference Style/Chemistry Reference Style
%%\documentclass[pdflatex, sn-mathphys,Numbered]{sn-jnl}% Math and Physical Sciences Reference Style
%%\documentclass[sn-aps]{sn-jnl}% American Physical Society (APS) Reference Style
%%\documentclass[sn-vancouver,Numbered]{sn-jnl}% Vancouver Reference Style
%%\documentclass[sn-apa]{sn-jnl}% APA Reference Style 
\documentclass[sn-chicago]{sn-jnl}% Chicago-based Humanities Reference Style
%%\documentclass[default]{sn-jnl}% Default
%%\documentclass[default,iicol]{sn-jnl}% Default with double column layout

%%%% Standard Packages
%%<additional latex packages if required can be included here>

\usepackage{graphicx}%
\usepackage{multirow}%
\usepackage{amsmath,amssymb,amsfonts}%
\usepackage{amsthm}%
\usepackage{mathrsfs}%
\usepackage[title]{appendix}%
\usepackage{xcolor}%
\usepackage{textcomp}%
\usepackage{manyfoot}%
\usepackage{booktabs}%
\usepackage{algorithm}%
\usepackage{algorithmicx}%
\usepackage{algpseudocode}%
\usepackage{listings}%
\usepackage{threeparttable}
\usepackage{dcolumn} %for stargazer in R
%%%%

% Hypersetup for hyperlinks
\usepackage{hyperref}
\hypersetup{
	colorlinks=true,            
	linkcolor={red!50!black},
	citecolor={blue!50!black},
	urlcolor={blue!80!black}
}
\usepackage{float}

%%%%%=============================================================================%%%%
% color scheme
\newcommand{\red}[1]{\textcolor{red}{#1}}
\newcommand{\blue}[1]{\textcolor{blue}{#1}}
\newcommand{\green}[1]{\textcolor{green}{#1}}
\newcommand{\teal}[1]{\textcolor{teal}{#1}}

%%%%%=============================================================================%%%%

%\jyear{2021}%

%% as per the requirement new theorem styles can be included as shown below
\theoremstyle{thmstyleone}%
\newtheorem{theorem}{Theorem}%  meant for continuous numbers
%%\newtheorem{theorem}{Theorem}[section]% meant for sectionwise numbers
%% optional argument [theorem] produces theorem numbering sequence instead of independent numbers for Proposition
\newtheorem{proposition}[theorem]{Proposition}% 
%%\newtheorem{proposition}{Proposition}% to get separate numbers for theorem and proposition etc.

\theoremstyle{thmstyletwo}%
\newtheorem{example}{Example}%
\newtheorem{remark}{Remark}%
\newtheorem{lemma}{Lemma}
\newtheorem{assumption}{Assumption}


\theoremstyle{thmstylethree}%
\newtheorem{definition}{Definition}%

\raggedbottom
%%\unnumbered% uncomment this for unnumbered level heads



\begin{document}
	
	\title[Article Title]{title}
	
	%%=============================================================%%
	%% Prefix	-> \pfx{Dr}
	%% GivenName	-> \fnm{Joergen W.}
	%% Particle	-> \spfx{van der} -> surname prefix
	%% FamilyName	-> \sur{Ploeg}
	%% Suffix	-> \sfx{IV}
	%% NatureName	-> \tanm{Poet Laureate} -> Title after name
	%% Degrees	-> \dgr{MSc, PhD}
	%% \author*[1,2]{\pfx{Dr} \fnm{Joergen W.} \spfx{van der} \sur{Ploeg} \sfx{IV} \tanm{Poet Laureate} 
		%%                 \dgr{MSc, PhD}}\email{iauthor@gmail.com}
	%%=============================================================%%
	
	
	
	\author*[1]{\fnm{Quang-Thanh} \sur{Tran}} \email{tran.quang.thanh.p1@dc.tohoku.ac.jp}
	
	%\equalcont{These authors contributed equally to this work.}
	
	\affil*[1]{\orgdiv{Graduate School of Economics and Management}, \orgname{Tohoku University}, \orgaddress{\street{41 Kawauchi}, \city{Sendai}, \postcode{980-0862}, \state{Miyagi}, \country{Japan}}}
	
	
	%%==================================%%
	%% sample for unstructured abstract %%
	%%==================================%%
	
	\abstract{hehe}
	
	
	\keywords{s}
	
	\pacs[JEL Classification]{E17, E23, I24, J13, J24, O31}
	
	%%\pacs[MSC Classification]{35A01, 65L10, 65L12, 65L20, 65L70}
	
	\maketitle
	
	\section{Introduction}
	
	Recent developments in automation technologies have transformed the production process and how people work. Machines are now replacing many repetitive tasks in manufacturing factories  \citep{schwab2017fourth}. The society also has fewer people performing those tasks, necessitating the R\&D in industrial robots that can replace humans \citep{acemoglu_demographics_2022}. In many industrialized countries, the increased stock of robots has moved in sync with the decrease in fertility rate, as depicted in Fig. \ref{fig:world_fertility}. We believe this trend results from a reciprocal relationship between machines and fertility. In particular, in order to help their children win the race against the machines, parents tend to increase spending on education , which would be more affordable by having fewer babies. As a consequence, the economy experience less human labor, which necessitate the demand for more machines in production. The rise in education spending also increases the proportion of high-skilled labor, which can be used to create smarter robots with wider range of variety. 
	

	
	\newpage
	\bibliography{sn-bibliography.bib}% common bib file
	%% if required, the content of .bbl file can be included here once bbl is generated
	%%\input sn-article.bbl
	
	\newpage
	\begin{appendices}
		
		\section{ Regression Results}
		
		\begin{table}[!htbp] \centering   \caption{Correlation Testing for Figure \ref{fig:jp_empirics}}   \label{table:regression} \begin{tabular}{@{\extracolsep{5pt}}lD{.}{.}{-3} D{.}{.}{-3} } \\[-1.8ex]\hline \hline \\[-1.8ex]  & \multicolumn{2}{c}{\textit{Dependent variable:}} \\  \\[-1.8ex] & \multicolumn{1}{c}{birth\_rate} & \multicolumn{1}{c}{skill\_premium} \\ \\[-1.8ex] & \multicolumn{1}{c}{(1)} & \multicolumn{1}{c}{(2)}\\ \hline \\[-1.8ex]  $\log$(robots\_per\_worker) & -1.165^{***} &  \\   & (0.085) &  \\   & & \\  $\log$ income & -1.228^{**} &  \\   & (0.461) &  \\   & & \\  enrollment\_rate & -4.580^{***} &  \\   & (0.678) &  \\   & & \\  $std\_robots\_per\_worker^2$ &  & 0.075^{***} \\   &  & (0.006) \\   & & \\  $std\_robots\_per\_worker$ &  & -0.295^{***} \\   &  & (0.018) \\   & & \\  Constant & 38.653^{***} & 1.670^{***} \\   & (5.580) & (0.012) \\   & & \\ \hline \\[-1.8ex] Observations & \multicolumn{1}{c}{36} & \multicolumn{1}{c}{36} \\ R$^{2}$ & \multicolumn{1}{c}{0.993} & \multicolumn{1}{c}{0.917} \\ Adjusted R$^{2}$ & \multicolumn{1}{c}{0.993} & \multicolumn{1}{c}{0.912} \\ Residual Std. Error & \multicolumn{1}{c}{0.145 (df = 32)} & \multicolumn{1}{c}{0.027 (df = 33)} \\ F Statistic & \multicolumn{1}{c}{1,567.013$^{***}$ (df = 3; 32)} & \multicolumn{1}{c}{181.539$^{***}$ (df = 2; 33)} \\ \hline \hline \\[-1.8ex] \textit{Note:}  &  \multicolumn{2}{r}{  $^{*}$p$<$0.1; $^{**}$p$<$0.05; $^{***}$p$<$0.01} \\
				& \multicolumn{2}{l}{ $std\_robots\_per\_worker$ is the positive standardization of robot density.} \\ \end{tabular} \end{table} 
		
	\end{appendices}
	
	%%===========================================================================================%%
	%% If you are submitting to one of the Nature Portfolio journals, using the eJP submission   %%
	%% system, please include the references within the manuscript file itself. You may do this  %%
	%% by copying the reference list from your .bbl file, paste it into the main manuscript .tex %%
	%% file, and delete the associated \verb+\bibliography+ commands.                            %%
	%%===========================================================================================%%
	
	
	
	
\end{document}